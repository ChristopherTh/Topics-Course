\documentclass[hyperref={pdfpagelabels=false}]{beamer}
\usepackage{lmodern}
\usepackage{amsmath}
\usepackage{graphicx}
\title{Explaining and Forecasting Online Auction
Prices and Their Dynamics Using
Functional Data Analysis}   
\author{Christopher Thiemann genannt Trappmann} 
\date{January 29, 2019} 

\begin{document}

\begin{frame}
\titlepage
\end{frame}

\begin{frame}{Introduction} %feel free to ask questions ?
\begin{itemize}
	\item Why did I choose the paper? %"Interested in how to apply the methods we leanred here, some background in auction theory here in my bachelöors"
	\item Here focus on the (functional) methods beeing used.
\end{itemize}
\end{frame}

\begin{frame}{Overview}
\begin{itemize}
	\item Why auctions? %"name some applications" auctions pupular %name examples
	\item Why dynamic functional forcasting model?
	\begin{itemize}
		\item deal with unevenly spaced observations
		\item capture price dynamics
	\end{itemize}	
\end{itemize}
overall structure of the paper
\begin{itemize}
	\item use Functional regression to explore price evolution and velocity
	\item use results to build the forecasting model
\end{itemize}
\end{frame}

\begin{frame}{Auction Framework}
\begin{itemize}
	\item ebay
	\item second price sealed bid with proxy bidding laaaaaaaaaaaaaaaaa
\end{itemize}
Data
\begin{itemize}
	\item 7 Day Auctions
	\item Microsoft Xbox and Harry Potter and the half blood prince
	\item Responds Variable: Bid History
\end{itemize}
\end{frame}

\begin{frame}{Bid-History} %erklärend bid spacing usw.
\begin{minipage}[c]{0.6\textwidth}
\includegraphics[width=\textwidth]{bid_history}
\end{minipage}
\hfill
\begin{minipage}[c]{0.3\textwidth}
\begin{itemize}
\item unevenly spaced observations
\item sparse in the middle
\item dense at the begining and end
\item bidsniping
\end{itemize}
\end{minipage}
\end{frame}

\begin{frame}{Auction Framework} %maybe not completelyy trivial if for examplke second highest bid bnelow reserve price
\begin{itemize}
	\item ebay
	\item second price sealed bid with proxy bidding
\end{itemize}
Data
\begin{itemize}
	\item 7 Day Auctions
	\item Microsoft Xbox and Harry Potter and the half blood prince
	\item Responds Variable: Bid History
	\item Explanatory variables: opening bid, final price, number of bids, seller rating, bidder rating, reserve price, condition, early bidding, jump bidding
\end{itemize}
\end{frame}

\begin{frame}{Functional Regression}
\begin{itemize}
	\item use FR to explore price dynamics
	\item relate functional responds variable to scalar covariates
	\item Modeltype: $\mathbf{y}(t)=\mathbf{X}^T\beta(t)+\mathbf{\epsilon}(t)$ %put emphasise on y(t) in comparison to the lecturre weher it was a scalar ,,,,,explain notation fett y ids vector of functions but i will talk about it in more detail later, if we plug in y'(t) we can model price velocity, critic no assumption on epsiloin?
	\item before however: recover functional data
\end{itemize}
\end{frame}

\begin{frame}{Irregular spacing of bids 1/3}
\centering
\begin{minipage}[c]{0.6\textwidth}
\includegraphics[width=\textwidth]{Auction1}
\end{minipage}
%\hfill
\begin{minipage}[c]{0.6\textwidth}
\includegraphics[width=\textwidth]{Auction2}
\end{minipage}
\end{frame}

\begin{frame}{Irregular spacing of bids 2/3}
%hier bild mit beiden log und linear interpoliert 
\end{frame}

\begin{frame}{Irregular spacing of bids 3/3}
%bild aus denenn von oben an einem common point gezogen wurde wie legt man diese puunkte ?? 
\end{frame}

\begin{frame}{Penalized smoothing spline}
\begin{align*}
\mathbf{y}^{(j)}=(y_1^{(j)},...,y_n^{(j)}) \ \ \ \ \ \	 j=1,...,N 
\end{align*}
 %"representation of each auction" 

polynomial spline
\begin{align*}
f(t)=\beta_0+\beta_2t+\beta_2t^2+...+\sum_{l=q}^L \beta_{pl}(t-\tau_l)^p_+ \\ 
\end{align*}
penalized smoothing spline
\begin{align*}
PENN^{(j)}_{\lambda,m}=\sum_{i=1}^n (y_i^{(j)}-f^{(j)}(t_i))^2+\lambda PEN^{(j)}_m \int_{}^{}D^mf(s)^2 ds %wie bestimmt man lambda und m und knots?
\end{align*}
\end{frame}

\begin{frame}{bild mit knots}

\end{frame}

\begin{frame}{bild mit verschiedene lkambdas 3 lambda gleich 0 genau richtig zu hoch}

\end{frame}

\begin{frame}{price dynamics}

\end{frame}

\begin{frame}
graph to graph
\end{frame}

\begin{frame}{Functional Regression Model}
\begin{align}
\mathbf{y}(t) = [y_1(t),...,y_N(t)] \nonumber \\ \mathbf{X}(t)=\mathbf{X} \nonumber \\ \mathbf{y}(t)=\mathbf{X}^T\beta(t)+\mathbf{\epsilon}(t) \nonumber
\end{align}
\end{frame}

\begin{frame}{Estimating the model}
\begin{itemize}
\item different approaches
\item fix a grid $t=t_1,...t_G$
\item for a fixed grid point, apply least squares
\item obtain estimates $\hat{\beta}(t_1),...,\hat{\beta}(t_G)$ 
\item smooth estimates to get functional estimate $\mathbf{\hat{\pmb{\beta}}}(t)$ 
\end{itemize}
\end{frame}

\begin{frame}{estimated paramjeter curves for price evolution} %muss interpretation zu jedem bild kennen nicht nur das was ich sagen will falls zb frage kommt % die variablen die ich in data erkläre hier interpretation zeigen!!!
\center
\includegraphics[width=0.7\textwidth]{price_evolution}
\end{frame}

\begin{frame}{estimated paramjeter curves for price velocity}
\center
\includegraphics[width=0.7\textwidth]{price_velocity}
\end{frame}

\begin{frame}{Dynamic forcasting model}
Four components
\begin{itemize}
	\item static predictor variables
	\item time-varying predictor variables
	\item price dynamics
	\item price lags
\end{itemize}	
\end{frame}

\begin{frame}{The model}
\begin{equation}
y(t|t-1)=\alpha+\sum_{i=1}^{Q}\beta_ix_i(t)+\sum_{j=1}^J\gamma_jD^{(j)}y(t|t-1)+\sum_{l=1}^L\eta_ly(t-l|t-l-1) \nonumber
\end{equation}
\newline \\
\begin{itemize}
	\item $x_1,...x_Q$ is the set of static and time variying predictors
	\item $D^{(j)}y(t|t-1)$ is the jth derivative
	\item $y(t-l|t-l-1)$ price lags
\end{itemize}
\end{frame}

\begin{frame}{h-step-ahead prediction}
\small
\begin{equation}
\tilde{y}(T+h|T)=\hat{\alpha}+\sum_{i=1}^{Q}\hat{\beta}_ix_i(T+h|T)+\sum_{j=1}^J\hat{\gamma}_j\hat{D}^{(j)}y(T+h|T)+\sum_{l=1}^L\hat{\eta}_l\tilde{y}(T+h-l|T) \nonumber
\end{equation}
\normalsize
\newline
Two challenges
\begin{itemize}
	\item at time $T \hat{D}^{(j)}y(T+h|T)$ is not known
	\item static predictors in $x_i$ do not change over time
\end{itemize}
\end{frame}

\begin{frame}{forecasting price dynamics}
We model $D^{(j)}y(t)$ as a polynomial in $t$ with autoregressive (AR) residuals
\begin{align}
D^{(j)}y(t|t-1) =\sum_{k=0}^Ka_kt^k+\sum_{i=1}^Pb_ix_i(t)+u(t) \ \ \ t=1,...,T \nonumber \\ \\ u(t)=\sum_{i=1}^R\phi u(t-i)+\epsilon(t) \ \ \ \ \epsilon(t) \sim iidN(0,\sigma^2)
\end{align} 
\end{frame}

\begin{frame}{forcasting price dynamics}
\begin{itemize}
	\item get estimates for all $a_k$, $b_i$ and $u(t)$
	\item use estimated residuals to estimate all $\phi_i$
	\item estimate next resdidual with \begin{equation}  \tilde{u}(T+1|T)=\sum_{i=1}^R\tilde{\phi}_iu(T-i+1) \nonumber \end{equation}
	\item lastly estimate the price derivative \begin{equation} D^{(j)}\tilde{y}(T+1|T) =\sum_{k=0}^K\hat{a}_k(T+1)^k+\sum_{i=1}^P\hat{b}_ix_i(T+1|T)+\tilde{u}(T+1|T) \nonumber \end{equation}
\end{itemize}	
\end{frame}

\begin{frame}{integrating static auction infromation}

\end{frame}

\begin{frame}

\end{frame}

\end{document}